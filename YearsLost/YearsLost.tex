%%This is a very basic article template.
%%There is just one section and two subsections.
\documentclass{article}

\usepackage{amsmath}
\usepackage{caption}
\usepackage{placeins}
\usepackage{graphicx}
\usepackage{subcaption}
\usepackage{tikz}
%\usepackage[active,tightpage]{preview}
\usepackage{natbib}
\bibpunct{(}{)}{,}{a}{}{;} 
\usepackage{url}
\usepackage{nth}
% for the d in integrals
\newcommand{\dd}{\; \mathrm{d}}
\newcommand{\tc}{\quad\quad\text{,}}
\newcommand{\tp}{\quad\quad\text{.}}

\begin{document}


\title{Years lived, years left and years lost}
\author{Tim Riffe}
\maketitle

%Dear A{\"i}da,
%The following is only a partly conceived idea nugget, possibly divisible into
%two or three separate papers. There is a part about mortality impacts on
%population structure, which can be separated; The main idea is on the
% demography of saved
%lives, with and without causes. Here you see some formulas and explanations
%(some figures attached to email), but I hope that the applications are
%imaginable and interesting enough, and that their place in the demographer's
%toolkit is evident. I'd be happy to receive your comment, suggestions, and if
% you're interested your signature! I imagine this as a PAA submission at first, but I think it can materialize quickly to be
%ready for a journal. Please let me know what you think!  Best wishes,
%Tim

\begin{abstract}
To structure a population by age is to represent its years already
lived, a reflection of past fertility and attrition. To structure a population
by remaining years of life is to represent its potential future; latency against
attrition. The years lost by a population due to death are equally viewable.
These lost years too can be classified by the ages in which the losses occurred, by the ages at which the lost years would have been lived, or by the
lost distribution of remaining years of life. Together these perspectives
characterize the timing, magnitude and interplay between length-of-life,
population stocks and attrition, distinguishing between past-future,
realized-foregone and their interactions.
\end{abstract}

Alternate titles: ``a demography of saved
lives'' or ``relationships between birth flows, death flows and population
stocks with more emphasis on the death part''

\section*{Getting on with it}
To begin, assume a closed population and ignore what determines the size of
successive birth cohorts, $B(t)$, which we usually call the birth flow. If
$B(t)$ is a given, then the only other factor that determines population size
and age structure is the length of the lives of its members. For simplicity,
assume that the age pattern to mortality is fixed (this isn't necessary, but it simplifies notation). The
population of age $a$ in year $t$, $P(a,t)$, is a function of the survival of the birth cohort born in year $t-a$. Since we're
fixing mortality, survival from age 0 to $a$ is $l(a)$ for any given cohort, so
$P(a,t) = B(t-a)l(a)$.

At any given $t$, there is a population stock that can be structured by birth
cohorts or age, the way we typically make population pyramids. If an entire
lifespan is denoted by the random variable, $X$, then the remaining lifespan,
$y$, of a still-alive person aged $a$, $y = X - a$. For the cohort born in year
$t-a$, the distribution of $X$ is given by $d(X)$, which is equal to the familar
lifetable $d_x$ for $x=X$, when calculated with a radix of 1.
The distribution of \textit{remaining} lifetimes for those having attained age
$a$, $f(y|a)$ is given by:
\begin{equation}
f(y|a) = \mu(a+y) \frac{l(a+y)}{l(a)} \tc
\end{equation}
which means that the population having survived to age $a$ and with $y$
remaining years of life is $P(a,y) = P(a)f(y|a)$ and the total population with
$y$ remaining years of life, $P(y)$, is simply $\int _{a=0}^\infty P(a,y) \dd
a$, a single death cohort with members from many birth cohorts. This
decomposition resorts the lifeline segments of a living population by the part
yet-unlived rather than by the part lived. The lived part we call ``age'' and
the unlived part has no common name. Since age is a sort of placemarker on the lifeline, I like to call both $a$ and $y$ ``age'', and differentiate by
calling them \textit{chronological} and \textit{thanatological}
age, respectively.\footnote{Thanatos was the Greek god of death, which marks the
end of the lifeline to which $y$ relates. By this token, one could just call chronological
age \textit{aphrodesian} age, but this would probably confuse things.}

The indices $a$ and $y$ differentiate between the past and future
parts of a lifeline, respectively, and by extension of populations when
so structured.
In comparing $P(a)$ with $P(y)$ we still refer to the living part of a population.
Is it so clear that the dead are no longer part of the population? If a life is
completely saved, this life stays in the living population and is not counted
as a death, but we think of saved lives as a transient state classification.
Perhaps this is not correct for demographers: amongst the living there
are no saved lives but only lived lives. Still we can quantify
hypothetically saveable life, and for this we must look to deaths.
It is nice (and often realistic) to think that many of the lives taken by death are (or will one day be) saveable, but little demography has been done of the lives that could be saved. If a saved life joins back into (so to speak) the homogeneous pool of alive people (the simplest way
to proceed), one may ask much more than how many lives can be saved, $D^s(a)$
(saveable lives, the number of real or would-be deaths at age $a$)
\begin{equation}
D^s(a) = \mu(a)P(a) \quad \quad \text{,}
\end{equation}
but also how many years would the saved people live? The simplest calculation is
to multiply the number of gained survivors by remaining life expectancy, $e(a)$:
\begin{align}
\label{eq:savedea}
D^s(a)e(a) &= P(a)\mu(a)\frac{1}{l(a)}\int_{y=0}^\infty l(a+y) \dd y \notag\\
         &= \mu(a)\int_{y=0}^\infty B(t-a)l(a+y) \dd y 
\end{align}
$D^s(a)e(a)$ assigns person-years gained by a population to the ages in which
they were saved. One may also wish
to know the distribution of remaining lifespans of saved lives, which is quite different than \eqref{eq:savedea}:
\begin{align}
\label{eq:savedy}
D^s(a)f(y|a) &= P(a)\mu(a)\mu(a+y) \frac{l(a+y)}{l(a)} \notag \tc\\
\intertext{and bringing things back to the birth flow for no apparent reason:}
           &= B(t-a)\mu(a)\mu(a+y)l(a+y) \tc
\end{align}
or indeed the ages in which the gained years of life would be lived, which is
the same as \eqref{eq:savedy} but shifted up element-wise by $a$. 

These basic relationships carry over when deaths and survival are adjusted to
account for the hypothetical elimination of particular causes (in the case of
exclusivity).
In this case $D^s$ is the sum of $c$ causes, and to speak of eliminating
cause $c$ from the lifetable is to speak of saving $D^{sc}$ lives and then
subject them to mortality conditions of all causes except $c$ (I give myself
carte blanche regarding competing risks for the time being).
This is problematic in that causes are not actually exclusive, and also in that we don't truly eliminate causes
entirely over all ages, but it serves as a good basis for comparing the relative
impacts of different causes on a given population structure. Similarity between
temporal patterns of saveable cause-specific remaining lifespans might also
serve as a basis for judging how much particular causes actually compete as
risks, wherein true ersatz causes will have identical distributions (and so may
be grouped).

Let us then define the force of mortality, $\mu(a) = \sum _{c=1}^n \mu^c(a)$, the sum of $n$ categorically separable causes. The people that will die from cause $c$ are:
\begin{align}
D^c &= \int_0^\infty D^c(a) \dd a \\
&= \int_0^\infty \mu^c(a)P(a) \dd a \tc\\
\intertext{and to hypothetically save all these people is to remove cause $c$
from mortality, retaining $D^c$ lives in the population. It makes sense to
calculate the distribution of remaining lifespans of the $D^c$ people that would
have died of this cause using $l(a)$ removed of the cause in question, so we define
$l^{-c}(a)$:}
l^{-c}(a) &= e^{-\int _0^a \mu(a)-\mu^c(a) \dd a} \tc\\
\intertext{which is hopefully more legible to render as:}
l^{-c}(a) &= e^{-\int _0^a \mu^{-c}(a) \dd a} \tp\\
\intertext{Continue with the same notational concept of ${ }^{-c}$ to define
remaining life expectancy assuming survival to age $a$ and no more death from
cause $c$ after age $a$, $e^{-c}(a)$:}
e^{-c}(a) &= \frac{1}{l^{-c}(a)}\int _y^\infty l^{-c}(a+y) \dd y \tp
\end{align}
and so on, repeating equations \eqref{eq:savedea} and \eqref{eq:savedy} for the
case of cause-specific saveable lives and their cause-deleted remaining
lifespans.

\section{Observed patterns}
The utility of the above relationships is best described
graphically at first, and perhaps more use examples can be computed. My first
visualizations are attached in the email (US 2010), but there are other
directions one could take.
I imagine a table of figures, perhaps comparing 6 casues for 4 countries (or
more, Magali may give access to the HMD COD beta collection if we ask nicely) or
time points (or something like that).
The hardest issue is making axes comparable\ldots 

%\section{Comments}
%Other ideas: we have production and consumption patterns over age from the NTA
%project, which would let the daring demographer assign values (cumulative,
% ouch!). I'll not brainstorm too much just yet unless more material is needed. I think the basic idea has enough
%traction on its own to not need to devise extentions just yet.

%If you think this idea is cool and worth pursuing I'd be happy to join forces
% as your time and interests allow. You said you're most interested in healthy life
%expectancy, but I'd like to think that this topic is an a substantial correlate
%and relevant for your research.

%For now, I'd cut out the birth flow age structure thing, that's just where the
%idea began\ldots

%I do not have a particular journal in mind, but I'm biased towards open access
%journals. Perhaps this would also make a nice PAA 2015 proposal (need to think
%about that already!).

\end{document}
