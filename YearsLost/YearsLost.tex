%%This is a very basic article template.
%%There is just one section and two subsections.
\documentclass{article}

\usepackage{amsmath}
\usepackage{caption}
\usepackage{placeins}
\usepackage{graphicx}
\usepackage{subcaption}
\usepackage{tikz}
%\usepackage[active,tightpage]{preview}
\usepackage{natbib}
\bibpunct{(}{)}{,}{a}{}{;} 
\usepackage{url}
\usepackage{nth}
\usepackage{authblk}
% for the d in integrals
\newcommand{\dd}{\; \mathrm{d}}
\newcommand{\tc}{\quad\quad\text{,}}
\newcommand{\tp}{\quad\quad\text{.}}
\defcitealias{HMD}{HMD}

\begin{document}


\title{Life lost, lifesaving, and causes of death.}

\author[1]{Tim Riffe\thanks{triffe@demog.berkeley.edu}}
\author[2]{A{\"i}da Sol\'{e} Aur\'{o}}
\affil[1]{Department of Demography, University of California, Berkeley}
\affil[2]{Institut National d'{\'E}tudes D{\'e}mographiques}

\maketitle

% AS: I know that you said that this paper is not motivated by a substantive
 % research question, but I
%feel that there is a lack of the main research point. Why are you doing this
% investigation? Are you going to present various descriptive possible scenarios about mortality 
% impacts and potentials gains (in years of good/bad life). To address so, the
% abstract needs to explain, in a very synthetic way, what the reader will find in the full manuscript.
%
\begin{abstract}
We classify cause of death impacts on the stock and structure of lives and
lifespans in ten countries using a counterfactual framework. The lives and potential years
of life lost due to death are presented as a metric for describing the population impacts of death and for comparing causes of
death. Lost lives and years of life may be classified by the ages in which
deaths occurred, by the ages to which deaths would be postponed were they saved, by the
ages in which the lost years would have been lived, or by the distribution of
lost remaining lifespans. These temporal perspectives define the
potential impacts of death and causes of death on population size and structure,
and on the distribution of lifespans within populations. 
\end{abstract}

%There is a need for an introduction. A general framework to situate the reader,
% which concludes with a research question. We also need a paragraph to present
% the data you are willing to use, as well as years and countries. The
% introduction might structure a little bit what comes next: A) Temporal
% relationships, B) Extension to causes of death; C) Population comparisons.
%
\section*{Introduction}
A core task of demography is to account for and predict the population
pyramid and the forces that shape it. The pyramid represents population
size and age structure, and it is shaped by the flows of births,
deaths, and migrations\footnote{We are guilty of omitting migration from
the following exposition, although some of the methods presented here would
translate cleanly to emigration.}. Of these flows, births are usually regarded
as the primary driver of variation in the profile of the pyramid,\footnote{This
is true to the extent that wars, epidemics, and other mortality shocks effect
broad age ranges rather than abrupt age groups.} whereas the pattern of survival
exterts a gentler influence on the overall shape of the pyramid.
Year to year variation in the number of deaths tends to deduct
smoothly from a wide range of ages, making mortality mostly invisible in the pyramid, whereas year to year variation in the number of births is preserved in ages. This is perhaps why 
relationships between mortality and age structure have been less charted than
those between births and age structure. 

Demographers most often quantify death in terms of rates 
because such quantities are considered purged of accidental distortions from
age structure. Public health institutes and news media often also report trends
in absolute numbers of deaths from particular causes and the total potential years of life lost (YPLL) due to
these deaths.\footnote{\citet{gardner1990} review commonly used methods of
calculating YPLL. The Global Burden of Disease reports refer to YPLL as YLL.
As a news media example, in 2013 the Guardian ran a data blog entry
visualizing the years of life lost in the USA due to gun violence
\citep{rogers2013gun}. } In either of these treatments, mortality and deaths are
isolated from age structure, although the notion of YPLL dovetails with
population size. In this paper we deviate from common practice, and treat deaths as population through the notion of potentially saveable
life. Demography based on this step is a counterfactual exercise, in
line with the formal treatment given in
\citet{vaupel1987repeated}.\footnote{\citet{vaupel2008lifesaving} follows in a
similar vein, but aims at the effects of lifesaving on period distortions.} We
will show that saveable life and lifetime can be quantified flexibly with respect to demographic perspectives on age and lifespan. These perspectives refine and supplement YPLL when assessing the population impacts of causes of death, and we
think that they may provide useful information for the targetting and planning
of public health interventions and the comparison of mortality burdens between
populations and subpopulations. The visualizations presented here may also
enrich standard reporting on the absolute and relative population impacts of
causes of death.

We first formally define what is meant by age and lifespan
perspectives, illustrating on the example of all-cause
mortality before proposing an extention to causes of death. Concepts are
illustrated based on the population of the United States in 2010. We then
compare a selection of five contemporary national populations (United States, England \& Wales, Norway, Sweden and Canada) based on pre-release
data newly collected by the Human Mortality
Database~\citepalias{HMD}.\footnote{Possibly up to ten poulations will be
covered if the data are compiled and ready prior to PAA. These would likely
include Japan, the Czech Republic, Belgium, France and Switzerland.} We propose
a few different ways of arranging and visualizing results for purposes of reporting or making comparisons. Finally we discuss the limits of these methods and the utility of the information gained by them. All data used in the following comes from the HMD.

\section*{Temporal relationships}
% To begin, sume a closed population and ignore what determines the size of
% successive birth cohorts, $B(t)$, which we usually call the birth flow. If
% $B(t)$ is a given, then the only other factor that determines population size
% and age structure is the length of the lives of its members. For simplicity,
% assume that the age pattern to mortality is fixed (this isn't necessary, but
% it simplifies notation). The population of age $a$ in year $t$, $P(a,t)$, is a function of 
% the survival of the birth cohort born in year $t-a$. Since we're
% fixing mortality, survival from age 0 to $a$ is $l(a)$ for any given cohort,
% so $P(a,t) = B(t-a)l(a)$.

% AS: Temporal relationships: I really like this section, but I think it needs
% to be reordered. Starting from Figure 1, moving to Figure 2 which perfectly illustrates what you 
% can do (using different populations) to compute the average years of life
% left. But I suggest you to first describe the idea, what you have done at the
% bottom of page 2 and then say something like: "and this translates into the following 
% form"; therefore, you can add your formulas. Followed by Figure 3. Since I am
% writing this email as I am reading, I just realized that you need a paragraph in front to introduce
% what follows: a) lifetime; b) remaining years of life; c) potentially savable lives. I need to 
% think more about these figures, but the work is there.  

Population stock in a
 given year, $t$, can be structured by birth cohorts or age, the way we typically make 
 population pyramids. If an entire lifespan is denoted by the random variable
 $X$, then the remaining lifespan, $y$, of a still-alive person aged $a$, $y = X - a$.
\begin{figure}[h]
% Figure produced in R/LifeLine.R
\centering
	\caption{A lifeline, where chronological age (years lived) is indexed by $a$
	and thanatological age (years left) is indexed by $y$.}
	\label{fig:line}
	\includegraphics[scale=.8]{Figures/LifeLine.pdf}	
\end{figure}
Figure~\ref{fig:line} gives a schematic representation of this simple
relationship between age and the lifespan for a single person-life. The lived
part we call ``age'' and the yet-unlived part has no common name. Since age is a
placemarker on the lifeline, both $a$ and $y$ could be called ``age'', and we
can specify them as \textit{chronological} and \textit{thanatological} age,
respectively.\footnote{Thanatos was the Greek god of death, which marks the end of the lifeline to which $y$ relates. By this token, one could just call chronological age \textit{aphrodesian} age, but this would probably confuse things.}

For a cohort, the distribution of $X$ is given by $f(X)$, which is equal to the familar lifetable
$d(a)$ for $a = X$, when the lifetable is specified with a radix equal
to unity ($l(0)=1$).
The distribution of \textit{remaining} lifetimes for those having survived to age $a$, $f(y|a)$ is
given by:\footnote{Equation \eqref{eq:fya} is easily modifiable to account for
mortality schedules that change over time.}
\begin{equation}
\label{eq:fya}
f(y|a) = \mu(a+y) \frac{l(a+y)}{l(a)} \tc
\end{equation}
where $\mu(a)$ is the force of mortality at exact age $a$, and $l(a)$ is
the value of the survival function at exact age $a$, proportional to the
probability of surviving from birth to age $a$. Figure~\ref{fig:fya} shows selected
cross-sections of the $f(y|a)$ surface calculated from the 2010 US male period
lifetable \citepalias{HMD}. The area under each chronological age-conditioned curve is equal to one.
As an individual grows older (here in constant mortality), the central mass of the
curve approaches zero, moving one year down per year lived. The shape of the center of
the curve does not change much until after chronological age 60, where
conditional rescaling drives up death probabilities more and more. Upward
scaling continues beyond those ages shown here, with $y=0$ becoming the greatest
single value in all ages beyond the modal chronological age at death.

 \begin{figure}[h]
\centering
% Figure produced in R/fya.R
	\caption{US males, 2010, $f(y|a)$ for selected ages.*}
	\label{fig:fya}
	\includegraphics[scale=.8]{Figures/fya.pdf}	
	\caption*{*Note that $f(y|0) = d(a) = f(X)$.}
\end{figure}

$f(y|a)$ can be used to calculate the population having survived to age $a$ and
with $y$ remaining years of life as $P(a,y) = P(a)f(y|a)$, where the total
population with $y$ remaining years of life, $P(y)$, is simply $\int
_{a=0}^\infty P(a,y) \dd a$, a single death cohort with members from many birth
cohorts. This decomposition sorts the lifeline segments of a living population
by the part yet-unlived (years left) rather than by the part lived. 

The indices $a$ and $y$ differentiate between the past and future
parts of a lifeline, respectively, and by extension of populations when
so structured.
In comparing $P(a)$ with $P(y)$, as in \citet{brouard1986structure}, we still
refer to the living (lived or to-be-lived) part of a population.
Is it so clear that the dead are no longer part of the population? If a life is
completely saved, this life stays in the living population and is not counted
as a death, but we (in common thinking) often imagine saved lives as a transient
state classification.
Perhaps this is not correct for demographers: amongst the living there
are no saved lives but only lived lives. Still we can quantify
hypothetically saveable life, and for this we must look to deaths.
It is nice, and often realistic, to think that many of the lives taken by death
are or will one day be saveable, but it is difficult to know what mortality
rates would apply to a population of saved individuals. To start, we consider
the hypothetical population of lives saved a single time from death and subject to
the same mortality as the population at large.

Assume that all the deaths recorded in a year are saved and brought back to
life. One may ask much more than the number and age structure of these saved
lives, $D^s(a)$,
\begin{equation}
\label{eq:Dsa}
D^s(a) = \mu(a)P(a) \tc
\end{equation}
but also how many years the people saved at age $a$ would live, 
$W^s(a)$\footnote{A mnemonic for $W$ could be \textit{won} years. This is
essentially an age at death breakdown of YPLL.}? The simplest calculation is
to multiply the number of gained survivors by remaining life expectancy at each
age, $e(a)$:
\begin{align}
\label{eq:savedea}
W^s(a) = D^s(a)e(a) &= P(a)\mu(a)\frac{1}{l(a)}\int l(a+y) \dd y \tp
%\notag\\
         %&= \mu(a)\int_{y=0}^\infty B(t-a)l(a+y) \dd y 
\end{align}
$D^s(a)e(a)$ classifies potentially saved person-years by the
ages in which they were saved. One may also wish to know the distribution of remaining lifespans of saved
lives, which is quite different from \eqref{eq:savedea}:
\begin{equation}
\label{eq:savedya}
D^s(a)f(y|a) = P(a)\mu(a)\mu(a+y) \frac{l(a+y)}{l(a)} \tc
\end{equation}
which aggregates up to the thanatological age distribution of saved lives,
$D^s(y)$:
\begin{equation}
\label{eq:savedy}
D^s(y) = \int D^s(a)f(y|a) \dd a \tp
\end{equation}
%or the time-to-death distribution of the total years of life
%gained,
%\begin{equation}
%\label{eq:gainedy}
%W^s(y) = \int D^s(a)\frac{l(a+y)}{l(a)} \dd a \tc
%\end{equation}
Or one might ask through which chronological ages the gained years of life
would be lived, $W^s(a+y|a)$,
\begin{equation}
\label{eq:gaineday}
W^s(a+y|a) = D^s(a)\frac{l(a+y)}{l(a)} \tp
\end{equation}

Figure~\ref{fig:Day} (left) shows US 2010 period
deaths (all the lives that could be saved) by age and sex (males on the left,
females on the right). Over 1.23 million deaths were recorded for males
and females each in 2010 in the US. Deaths have been decomposed into discrete
categories of remaining years of life using equation~\eqref{eq:fya}, under the assumption that saved lives are subject to the same mortality schedule as the rest of the population, and under the supposition that all 2010 deaths get saved (just once). The results of this decompositon are represented by color bands in
Figure~\ref{fig:Day}. The average chronological age at death
for males was a full seven years lower than that for females
(69.9 versus, 76.9, respectively). Figure~\ref{fig:Dya} (right) displays the
same decomposition after swapping the y axis and color gradient from Figure~\ref{fig:Day}. Now thanatological age (years left) of hypothetically saved lives are the primary y axis, while chronological age groups (years lived) are displayed with color. Figure~\ref{fig:Dya} communicates that most saveable
lives would live short remaining lifespans once saved and granted the same lifetable
mortality. This is so because most saveable lives are already in high
chronological ages. In general, the only saveable lives that might live very
long remaining lifespans are the few deaths that occur in young ages. 

Randomly
selected saveable males from this population would have on average longer
remaining lifespans than randomly selected saveable females (16.7 versus 13.3 years,
respectively). This is a paradox because females have lower mortality rates in
nearly all ages, and have longer remaining life expectancies in all ages. Female
mortality advantage is in this case more than offset by the relative youth of
male deaths. Untangling the paradox further becomes a recursive exercise,
since the relative youth of male deaths is due to an interaction between
mortality schedules and population structure, iself an outcome of past vital forces.

\begin{figure}
\centering
\caption{US, 2010 potentially saveable lives (Deaths)}
\label{fig:1}
\begin{subfigure}[b]{.48\linewidth}
\centering
	\caption{Classified by age (years lived) and sex, and decomposed
by hypothetical remaining years of life (years left).}
	\label{fig:Day}
	% Figure produced in R/AllCauseFigures.R
	\includegraphics[scale=.55]{Figures/Deathsxy10.pdf}	
	\caption*{$D^s(a)$ from equation~\eqref{eq:Dsa}}
\end{subfigure}
~
\begin{subfigure}[b]{.48\linewidth}
\centering
    \caption{Classified by hypothetical remaining years of life
(years left) and sex, and decomposed by age (years lived).}
	\label{fig:Dya}
	% Figure produced in R/AllCauseFigures.R
    \includegraphics[scale=.55]{Figures/Deathsyx10.pdf}
    \caption*{$D^s(y)$ from equation~\eqref{eq:savedy}}
\end{subfigure}
\end{figure}

\begin{figure}
\centering
\caption{US, 2010 person years of life potentially won*}
\label{fig:2}
\begin{subfigure}[b]{.48\linewidth}
\centering
	\caption{Classified by age at hypothetical saving and sex, $W^s(a)$, and
	decomposed by future ages to be lived.}
	\label{fig:SavedGained}
	% Figure produced in R/AllCauseFigures.R
	\includegraphics[scale=.55]{Figures/YearsSavedGainedxx10.pdf}
	\caption*{$W^s(a)$ from equation~\ref{eq:savedea}}	
\end{subfigure}
~
\begin{subfigure}[b]{.48\linewidth}
\centering
    \caption{Classified by cumulative ages to be lived through and sex, and
    decomposed by age at saving.}
	\label{fig:LostLived}
	% Figure produced in R/AllCauseFigures.R
    \includegraphics[scale=.55]{Figures/YearsLostLivedyx10.pdf}
    \caption*{$W^s(a+y|a)$ from equation~\ref{eq:gaineday}}	
\end{subfigure}
\caption*{(Note different x scale from Figure~\ref{fig:1}).}
\end{figure}

Figure~\ref{fig:SavedGained} shows the results of applying
equation~\eqref{eq:savedea} to the same US data, which is essentially a
reweighting of Figure~\ref{fig:Day} by remaining life expectancy. Color bands
are assigned by decomposing the total life to be lived into the ages through
which it will be lived. For example, if we save all 11700 of the 50-year-old
males that died in 2010, they would live a total of 349000 combined years (under
constant and homogenous mortality), spread out over ages 50 and higher according
to $\frac{l(50+y)}{l(50)}$. In Figure~\ref{fig:SavedGained} we decompose these
gained years of life according to equation~\eqref{eq:savedy} (using $f(a+y|a)$)
and highlight this decomposition with color, while in Figure~\ref{fig:LostLived}, \textit{gained} ages become the primary y axis,
while color bands represent the ages in which populations in each age group were
saved. Figure~\ref{fig:LostLived} represents the cumulative contribution to the
population pyramid that would result from saving all lives in 2010 and then
surviving them forward according to 2010 mortality conditions.

\section*{Causes of death}
% AS: Extension to causes of death. This is the case you want to make. Saying,
% for instance, what if we eliminate mortality from heart disease? Or stroke? or hypertension? We 
% need to explain an story here and the draft needs to be shaped according to
% that. I think that all your mathematical material is very relevant but we need to get deeper into 
% the general idea and maybe leave the formulas for an appendix. But it is ok
% for the moment, we will think about that later.

These basic relationships carry over when deaths and survival are adjusted to
account for the hypothetical elimination of particular causes (in the case of
independence). In this case $D^s$ is the sum of $n$ causes, and to speak of
eliminating cause $c$ from the lifetable is to speak of saving $D^{sc}$ lives and then
subject them to the mortality conditions of all causes except $c$.
This is problematic in that causes are not independent, and also in that
reductions in cause-specific mortality are not so thorough , but it serves as a
good basis for comparing the relative impacts of different causes on a given population structure.

Let us then define the force of mortality, $\mu(a) = \sum _{c=1}^n \mu^c(a)$,
as the sum of $n$ categorically separable causes. The people that will die from
cause $c$ are:
\begin{align}
D^c &= \int_0^\infty D^c(a) \dd a \\
&= \int_0^\infty \mu^c(a)P(a) \dd a \tc\\
\intertext{and to hypothetically save all these people is to remove cause $c$
from mortality, retaining $D^c$ lives in the population. It makes sense to
calculate the distribution of remaining lifespans of the $D^c$ people that would
have died of this cause using $l(a)$ removed of the cause in question, so we define
$l^{-c}(a)$,}
l^{-c}(a) &= e^{-\int _0^a \mu(a)-\mu^c(a) \dd a} \tc\\
\intertext{which is hopefully more legible to render as}
l^{-c}(a) &= e^{-\int _0^a \mu^{-c}(a) \dd a} \tp\\
\intertext{This is a stronger supposition than the idea of repeated
resuscitation from \citet{vaupel1987repeated}, but the idea is to separate out
the impacts of particular causes.
Let us continue with the same notational concept of ${ }^{-c}$ to define
remaining life expectancy assuming survival to age $a$ and no more death from cause $c$ after age $a$, $e^{-c}(a)$:} e^{-c}(a) &= \frac{1}{l^{-c}(a)}\int _{y=0}^\infty l^{-c}(a+y) \dd y \tp
\end{align}
and so on, repeating equations \eqref{eq:savedea} and \eqref{eq:savedy} for the
case of cause-specific saveable lives and their cause-deleted remaining
lifespans.

Now Figures~\ref{fig:1} and~\ref{fig:2} can be repeated for
any particular cause of death, and the profile of each of the four perspectives
characterizes the population impact of the given cause of death.
Figures~\ref{fig:3} and~ \ref{fig:4} depict the same temporal viewpoints,
respectively, but now the deaths decomposed are only deaths to external
causes, and external causes have been eliminated from the lifetable functions used for decomposition and
 redistribution. Causes differ in their impact profiles, and this can form a
 basis for comparison. As with population pyramids, one may prefer the use of
 percent scales to facilitate comparisons between causes or countries. These
 figures for external causes give visual form to the intuition that demographers
 aleady have about external causes of death. External causes are important in
 young ages, and affect males more than females. Saving a randomly selected
 death from an external cause will on average have a higher payoff in terms of
 expected years of life gained than does preventing a death in general. Further,
 a typical life saved will traverse many working ages, and reach well into old
 ages.
 The group with the most to gain by eliminating external causes are males in
 their 20s.
 
 The average chronological age of deaths to external causes in the USA in 2010
 was 47.9 for males and 57.1 for females. Their cause-deleted mean remaining
 lifetimes were 34.0 and 29.8 years, respectively. The mean age-at-saving of all
 the person-years hypothetically won under these same conditions becomes 36.7
 for males and 40.3 for females, whereas the mean of the ages \textit{enjoyed}
 by these hypothetically saved people are 59.7 and 63.3, respectively. Means do
 not tell the story as well as images. 

\begin{figure}
\centering
\caption{USA, 2010 Deaths from external causes}
\label{fig:3}
\begin{subfigure}[b]{.48\linewidth}
\centering
	\caption{Classified by age (years lived) and sex, and decomposed
by hypothetical remaining years of life (years left).}
	\label{fig:Dayc}
	% Figure produced in R/SingleCauseExampleFigures.R
	\includegraphics[scale=.55]{Figures/Deathsxy10USAExternal.pdf}	
	\caption*{$D^{sc}(a)$}
\end{subfigure}
~
\begin{subfigure}[b]{.48\linewidth}
\centering
    \caption{Classified by hypothetical remaining years of life
(years left) and sex, and decomposed by age (years lived).}
	\label{fig:Dyac}
	% Figure produced in R/SingleCauseExampleFigures.R
    \includegraphics[scale=.55]{Figures/Deathsyx10USAExternal.pdf}
    \caption*{$D^{sc}(y)$ }
\end{subfigure}
\end{figure}


\begin{figure}
\centering
\caption{USA, 2010 Deaths from external causes, years of life potentially won*}
\label{fig:4}
\begin{subfigure}[b]{.48\linewidth}
\centering
	\caption{Classified by age at hypothetical saving and sex, $W^s(a)$, and
	decomposed by future ages to be lived.}
	\label{fig:SavedGainedUSAExternal}
	% Figure produced in R/SingleCauseExampleFigures.R
	\includegraphics[scale=.55]{Figures/YearsSavedGainedxx10USAExternal.pdf}
	\caption*{$W^s(a)$ from equation~\ref{eq:savedea}}	
\end{subfigure}
~
\begin{subfigure}[b]{.48\linewidth}
\centering
    \caption{Classified by cumulative ages to be lived through and sex, and
    decomposed by age at saving.}
	\label{fig:LostLivedUSAExternal}
	% Figure produced in R/SingleCauseExampleFigures.R
    \includegraphics[scale=.55]{Figures/YearsLostLivedyx10USAExternal.pdf}
    \caption*{$W^s(a+y|a)$ from equation~\ref{eq:gaineday}}	
\end{subfigure}
\caption*{(Note different x scale from Figure~\ref{fig:3}).}
\end{figure}

\FloatBarrier

\section*{Observed patterns}
In this section we summarize cause of death impacts from eight
major cause groups: cancers, cardiovascular disease, external causes, ill-defined casues, infant and congenital causes, infectious disease, mental
debilitation, and other causes. These results will be summarized for the USA,
Canada, England and Wales, Norway, and Sweden based on a soon-to-be
released collection from the HMD. This section may have a supplemental appendix
of results.

\section*{Discussion}

%---------------------------------------


%Then, decide whether to present a separate panel for each country of all causes
%for each perspective, or rather each cause in a unique panel comparing
%countries. I'm leaning toward each each country in a separate panel, but need
% to think on it more. I'm doubting whether it makes sense to compare populations
%that are not stable, since part of differences will be due to previous
%population structure. Comparing causes (save for competing risks) works within
% a population because it has a single structure, but between countries it may only
%make sense to go back to rates, but then we lose the point of the paper. This
% is my tentative stance on between-country comparisons. It's of course all easy to
%decompose, but that seems like a frivolous step.
% AS: Regarding expanding the analysis to more populations, I think that you are
% right, but part of the differences would be not only due to the structure but the context 
% of each population.
% 
---------------------------------------
\bibliographystyle{plainnat}
  \bibliography{references}  
\end{document}
